\section{Independent Models}

\begin{frame}{Statistical Models}
    \begin{block}{planetmath.org}
    A \emph{statistical model} is usually parameterised by a function, called a \emph{parametrisation}
    $$ \Theta \ra \mc{P}, \quad \text{given by} \quad \theta \mapsto P_{\theta},\quad \text{so that}\quad \mc{P} = \{ P_{\theta} : \theta \in \Theta \}, $$
    where $\Theta$ is the \emph{parameter space}. $\Theta$ is usually a subset of $\RR^{n}$.
    \end{block}

    \begin{block}{McCullagh, 2002}
    This should be defined using category theory.
    \end{block}
\end{frame}

\begin{frame}{Thee-Way Contingency Tbales}
    Let $X, Y$, and $Z$ be random variables that have $a, b$, and $c$ states, respectively. A \emph{probability distribution} $P$ for these random variables is an $(a \times b \times c)$-table of non-negative numbers which sum to one.

    The entries of the table $P$ are the probabilities
    $$ P_{ijk} = \Prob(X = i; Y = j; Z = k). $$
    The set of all distributions is a simplex $\Delta$ of dimension $abc - 1$.

    A \emph{statistical model} is a subset $\mc{M}$ of $\Delta$ which can be described by polynomial equations and inequalities in the coordinates $P_{ijk}$.

    Usually, the model $\mc{M}$ is presented as the image of a polynomial map $P: \Theta \ra \Delta$, where $\Theta$ is a polynomially described subset of $\Delta$.
\end{frame}

\begin{frame}{Independence}
    The distribution $P$ is called \emph{independent} is each probability is the product of the corresponding \emph{marginal probabilities}:
    $$ P_{ijk} = P_{i++} \cdot P_{+j+} \cdot P_{++k}. $$
    A marginal probability is the probability of an event irrespective of the outcomes of the other variables, that is:
    $$ P_{i++} = \Prob(X = i) = \sum_{j=1}^{b} \sum_{k=1}^{c} P_{ijk}. $$
\end{frame}

\begin{frame}{Independence Model}
    The \emph{independence model} has the parametric representation:
    \begin{equation*}
        \begin{split}
            \Theta = \Delta_{a-1} \times \Delta_{b-c} \times \Delta_{c-1} &\ra \Delta = \Delta_{abc - 1}, \\
            (\alpha,\beta,\gamma) \mapsto (P_{ijk}) = (\alpha_{i}\beta_{j}\gamma_{k}).
        \end{split}
    \end{equation*}

    The image is known as the \emph{Segre variety} in algebraic geometry.
\end{frame}

\begin{frame}{Foray Into Algebraic Geometry}
    \begin{block}{Projective Space}
    Playing field is $n$\emph{-dimensional projective space}, $\PP^{n}$:
        $$ \PP^{n} := \{ (z_{0}, \ldots, z_{n}) \in \CC^{n} \} / (\vb{x} \sim \lambda \cdot \vb{y}), \quad \lambda \neq 0, $$
    that is, its elements consists of \emph{lines through the origin} in $\CC^{n}$.
    \end{block}
    
    TODO: FIGURE
\end{frame}

\begin{frame}{Varieties}
        \emph{Varieties} are the geometric objects studied in algebraic\footnote{classical algebraic geometry} geometry, which are the \emph{vanishing sets} for polynomials.

        Example:
        $$ S^{1} = \{ x^{2} + y^{2} = 1 \} = V( x^{2} + y^{2} - 1 ). $$

        The $V$ stands for \emph{Verschwindungsmenge}, meaning vanishing set.

    TODO: FIGURE
\end{frame}

\begin{frame}{Segre Varieties}
        These come from the maps $\sigma: \PP^{n} \times \PP^{m} \ra \PP^{(n+1)(m+1) - 1}$, sending the pair $([X],[Y])$ to the coordinates formed by pairwise products of the individual $[X]$ and $[Y]$:
            $$ \sigma : ([X_{0}, \ldots, X_{n}], [Y_{0}, \ldots, Y_{m}]) \mapsto [\ldots, X_{i}Y_{j}, \ldots ], $$
        with the image is formed overall pairwise products of $X_{i}$ and $Y_{j}$.

        \begin{block}{Example}
        $$\sigma : \PP^{1} \times \PP^{1} \ra \PP^{3},\ ([X_{0}, X_{1}], [Y_{0}, Y_{1}]) \mapsto [ X_{0}Y_{0}, X_{0}Y_{1}, X_{1}Y_{0}, X_{1}Y_{1} ].  $$
        
        If we set $[ X_{0}Y_{0}, X_{0}Y_{1}, X_{1}Y_{0}, X_{1}Y_{1} ] = [Z_{0}, Z_{1}, Z_{2}, Z_{3}]$, then notice that $Z_{0}Z_{3} - Z_{1}Z_{2} = 0$.
        \end{block}
\end{frame}