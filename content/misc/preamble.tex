%
% XeTeX specific
%
\RequireXeTeX
\usepackage{xltxtra}
\usepackage{fontspec}
\usepackage{xunicode}

% -----------------------------------
% font config
% -----------------------------------

\setsansfont{Linux Biolinum}
\setromanfont{Linux Libertine}
\setmonofont[Scale=0.9]{Consolas}

% -----------------------------------
% math
% -----------------------------------

\usepackage{amsmath}
\usepackage{amsfonts}
\usepackage{amssymb}
\usepackage{tikz}
\usetikzlibrary{decorations.pathmorphing}
\usepackage{amsthm}
\usepackage{mathrsfs}   % \mathscr
\usepackage{stmaryrd}   % \lightning
\usepackage{physics}
\usepackage{booktabs}
\usepackage{pgfplots}
\pgfplotsset{compat=1.17}
\usepackage{pgfplotstable}

% -----------------------------------
% macros
% -----------------------------------

\newcommand{\ra}{\rightarrow}
\newcommand{\la}{\leftarrow}
\newcommand{\lra}{\longrightarrow}
\newcommand{\lla}{\longleftarrow}
\newcommand{\into}{\hookrightarrow}
\newcommand{\NN}{\mathbb{N}}
\newcommand{\ZZ}{\mathbb{Z}}
\newcommand{\QQ}{\mathbb{Q}}
\newcommand{\RR}{\mathbb{R}}
\newcommand{\CC}{\mathbb{C}}
\newcommand{\PP}{\mathbb{P}}
\newcommand{\half}{\frac{1}{2}}
\newcommand{\Mat}{\text{Mat}}
\newcommand{\Span}{\text{Span}}
\newcommand{\Id}{\operatorname{Id}}
\newcommand{\Td}{\operatorname{Td}}
\newcommand{\Prob}{\operatorname{Prob}}
\newcommand{\Mixt}{\operatorname{Mixt}}
\newcommand{\Sec}{\operatorname{Sec}}
\newcommand{\mc}[1]{\mathcal{#1}}

\newcommand{\indep}{\perp \!\!\! \perp}
\newlength{\LETTERheight}
\newcommand*{\longleadsto}[1]{\ \raisebox{0.24\LETTERheight}{\tikz \draw [->,
line join=round,
decorate, decoration={
    zigzag,
    segment length=4,
    amplitude=.9,
    post=lineto,
    post length=2pt
}] (0,0) -- (#1,0);}\ }

\newcommand{\createcontingencytable}[4]{ %
% #1=table name
% #2=first column name
% #3=new row sum name
% #4=new column sum name
\pgfplotstablecreatecol[
    create col/assign/.code={% In each row ... 
        \def\rowsum{0}
        \pgfmathtruncatemacro\maxcolindex{\pgfplotstablecols-1}
        % ... loop over all columns, summing up the elements
        \pgfplotsforeachungrouped \col in {1,...,\maxcolindex}{
            \pgfmathsetmacro\rowsum{\rowsum+\thisrowno{\col}}
        }
        \pgfkeyslet{/pgfplots/table/create col/next content}\rowsum
    }
]{#3}{#1}%
%
% Transpose the table, so we can repeat the summation step for the columns
\pgfplotstabletranspose[colnames from={#2},input colnames to={#2}]{\intermediatetable}{#1}
%
% Sums for each column
\pgfplotstablecreatecol[
    create col/assign/.code={%
        \def\colsum{0}
        \pgfmathtruncatemacro\maxcolindex{\pgfplotstablecols-1}
        \pgfplotsforeachungrouped \col in {1,...,\maxcolindex}{
            \pgfmathsetmacro\colsum{\colsum+\thisrowno{\col}}
        }
        \pgfkeyslet{/pgfplots/table/create col/next content}\colsum
    }
]{#4}\intermediatetable
%
% Transpose back to the original form
\pgfplotstabletranspose[colnames from=#2, input colnames to=#2]{\contingencytable}{\intermediatetable}
}
%

% -----------------------------------
% grammar and textstyle
% -----------------------------------

\usepackage{polyglossia}
\setdefaultlanguage[variant=british]{english}
\usepackage{csquotes}
% underlining
\usepackage{soul} % ulem redifines \emph, this sucks

% -----------------------------------
% colors
% -----------------------------------

\usepackage{xcolor}
\usepackage{colortbl}

% listing colors
% https://kuler.adobe.com/Color-palette-color-theme-4004722
\definecolor{keyword}{RGB}{239,33,74}
\definecolor{number}{RGB}{243,202,22}
\definecolor{comment}{RGB}{126,158,19}
\definecolor{string}{RGB}{6,129,128}


% -----------------------------------
% links and references
% -----------------------------------

\usepackage{hyperref}
\hypersetup{
    colorlinks=false,
    % hide bookmarks
    pdfpagemode=UseNone,
    % meta
    pdfauthor={\presentationAuthor},
    pdftitle={\presentationTitle}
}
\usepackage[english]{cleveref}

% -----------------------------------
% bibliography and glossaries
% -----------------------------------

\usepackage[
    backend=biber,
    style=alphabetic-verb
    ]{biblatex}
\bibliography{presentation}

\usepackage{glossaries}
\input{glossary}
\makeglossaries

% -----------------------------------
% graphics
% -----------------------------------

\usepackage{graphicx}
\usepackage{tikz}
\usetikzlibrary{shapes.geometric, arrows, shadows, positioning}
\usepackage{adforn} % ornaments, used in titlepage
\usepackage{caption}
\usepackage{subcaption}

% -----------------------------------
% beamer theme
% -----------------------------------

% you can't locate the theme in a subfolder without shooting yourself in the knee
\usetheme{alinz}

% -----------------------------------
% listings and pseudocode
% -----------------------------------

\Crefname{lstlisting}{Listing}{Listings}
\crefname{lstlisting}{listing}{listings}

\usepackage{listings}
\lstset{
    basicstyle=\footnotesize\ttfamily\color{lightdark},
    backgroundcolor=\color{blockbg},
    numbers=left,
    %numbersep=6pt,
    numberstyle=\scriptsize\color{granite},
    commentstyle=\sffamily\itshape\color{sea},
    keywordstyle=\bfseries\color{raspberry},
    stringstyle=\itshape\color{lake},
    showstringspaces=false,
    breaklines=true,
    breakatwhitespace=true,
    frame=lr,
    framerule=0pt,
    framesep=6pt,
    captionpos=b
}
% for pseudocode
\usepackage[slide,linesnumbered,algoruled]{algorithm2e}

%% http://tex.stackexchange.com/questions/99500/listings-code-style-for-html5-css-html-javascript
\lstdefinelanguage{JavaScript}{
    morekeywords={break, case, catch, continue, debugger, default, delete, do, else, false, finally, for, function, if, in, instanceof, new, null, return, switch, this, throw, true, try, typeof, var, void, while, with},
    morecomment=[s]{/*}{*/},
    morecomment=[l]//,
    morestring=[b]",
    morestring=[b]'
}

%% http://tex.stackexchange.com/questions/99500/listings-code-style-for-html5-css-html-javascript
\lstdefinelanguage{HTML5}{
        language=html,
        sensitive=true,
        alsoletter={<>=-},
        otherkeywords={
        % HTML tags
        < </, >,
        </a, <a, </a>,
        </abbr, <abbr, </abbr>,
        </address, <address, </address>,
        </area, <area, </area>,
        </area, <area, </area>,
        </article, <article, </article>,
        </aside, <aside, </aside>,
        </audio, <audio, </audio>,
        </audio, <audio, </audio>,
        </b, <b, </b>,
        </base, <base, </base>,
        </bdi, <bdi, </bdi>,
        </bdo, <bdo, </bdo>,
        </blockquote, <blockquote, </blockquote>,
        </body, <body, </body>,
        </br, <br, </br>,
        </button, <button, </button>,
        </canvas, <canvas, </canvas>,
        </caption, <caption, </caption>,
        </cite, <cite, </cite>,
        </code, <code, </code>,
        </col, <col, </col>,
        </colgroup, <colgroup, </colgroup>,
        </data, <data, </data>,
        </datalist, <datalist, </datalist>,
        </dd, <dd, </dd>,
        </del, <del, </del>,
        </details, <details, </details>,
        </dfn, <dfn, </dfn>,
        </div, <div, </div>,
        </dl, <dl, </dl>,
        </dt, <dt, </dt>,
        </em, <em, </em>,
        </embed, <embed, </embed>,
        </fieldset, <fieldset, </fieldset>,
        </figcaption, <figcaption, </figcaption>,
        </figure, <figure, </figure>,
        </footer, <footer, </footer>,
        </form, <form, </form>,
        </h1, <h1, </h1>,
        </h2, <h2, </h2>,
        </h3, <h3, </h3>,
        </h4, <h4, </h4>,
        </h5, <h5, </h5>,
        </h6, <h6, </h6>,
        </head, <head, </head>,
        </header, <header, </header>,
        </hr, <hr, </hr>,
        </html, <html, </html>,
        </i, <i, </i>,
        </iframe, <iframe, </iframe>,
        </img, <img, </img>,
        </input, <input, </input>,
        </ins, <ins, </ins>,
        </kbd, <kbd, </kbd>,
        </keygen, <keygen, </keygen>,
        </label, <label, </label>,
        </legend, <legend, </legend>,
        </li, <li, </li>,
        </link, <link, </link>,
        </main, <main, </main>,
        </map, <map, </map>,
        </mark, <mark, </mark>,
        </math, <math, </math>,
        </menu, <menu, </menu>,
        </menuitem, <menuitem, </menuitem>,
        </meta, <meta, </meta>,
        </meter, <meter, </meter>,
        </nav, <nav, </nav>,
        </noscript, <noscript, </noscript>,
        </object, <object, </object>,
        </ol, <ol, </ol>,
        </optgroup, <optgroup, </optgroup>,
        </option, <option, </option>,
        </output, <output, </output>,
        </p, <p, </p>,
        </param, <param, </param>,
        </pre, <pre, </pre>,
        </progress, <progress, </progress>,
        </q, <q, </q>,
        </rp, <rp, </rp>,
        </rt, <rt, </rt>,
        </ruby, <ruby, </ruby>,
        </s, <s, </s>,
        </samp, <samp, </samp>,
        </script, <script, </script>,
        </section, <section, </section>,
        </select, <select, </select>,
        </small, <small, </small>,
        </source, <source, </source>,
        </span, <span, </span>,
        </strong, <strong, </strong>,
        </style, <style, </style>,
        </summary, <summary, </summary>,
        </sup, <sup, </sup>,
        </svg, <svg, </svg>,
        </table, <table, </table>,
        </tbody, <tbody, </tbody>,
        </td, <td, </td>,
        </template, <template, </template>,
        </textarea, <textarea, </textarea>,
        </tfoot, <tfoot, </tfoot>,
        </th, <th, </th>,
        </thead, <thead, </thead>,
        </time, <time, </time>,
        </title, <title, </title>,
        </tr, <tr, </tr>,
        </track, <track, </track>,
        </u, <u, </u>,
        </ul, <ul, </ul>,
        </var, <var, </var>,
        </video, <video, </video>,
        </wbr, <wbr, </wbr>,
        />, <!
        },
        ndkeywords={
        % General
        =,
        % HTML attributes
        accept=, accept-charset=, accesskey=, action=, align=, alt=, async=, autocomplete=, autofocus=, autoplay=, autosave=, bgcolor=, border=, buffered=, challenge=, charset=, checked=, cite=, class=, code=, codebase=, color=, cols=, colspan=, content=, contenteditable=, contextmenu=, controls=, coords=, data=, datetime=, default=, defer=, dir=, dirname=, disabled=, download=, draggable=, dropzone=, enctype=, for=, form=, formaction=, headers=, height=, hidden=, high=, href=, hreflang=, http-equiv=, icon=, id=, ismap=, itemprop=, keytype=, kind=, label=, lang=, language=, list=, loop=, low=, manifest=, max=, maxlength=, media=, method=, min=, multiple=, name=, novalidate=, open=, optimum=, pattern=, ping=, placeholder=, poster=, preload=, pubdate=, radiogroup=, readonly=, rel=, required=, reversed=, rows=, rowspan=, sandbox=, scope=, scoped=, seamless=, selected=, shape=, size=, sizes=, span=, spellcheck=, src=, srcdoc=, srclang=, start=, step=, style=, summary=, tabindex=, target=, title=, type=, usemap=, value=, width=, wrap=,
        % CSS properties
        accelerator:,azimuth:,background:,background-attachment:,
        background-color:,background-image:,background-position:,
        background-position-x:,background-position-y:,background-repeat:,
        behavior:,border:,border-bottom:,border-bottom-color:,
        border-bottom-style:,border-bottom-width:,border-collapse:,
        border-color:,border-left:,border-left-color:,border-left-style:,
        border-left-width:,border-right:,border-right-color:,
        border-right-style:,border-right-width:,border-spacing:,
        border-style:,border-top:,border-top-color:,border-top-style:,
        border-top-width:,border-width:,bottom:,caption-side:,clear:,
        clip:,color:,content:,counter-increment:,counter-reset:,cue:,
        cue-after:,cue-before:,cursor:,direction:,display:,elevation:,
        empty-cells:,filter:,float:,font:,font-family:,font-size:,
        font-size-adjust:,font-stretch:,font-style:,font-variant:,
        font-weight:,height:,ime-mode:,include-source:,
        layer-background-color:,layer-background-image:,layout-flow:,
        layout-grid:,layout-grid-char:,layout-grid-char-spacing:,
        layout-grid-line:,layout-grid-mode:,layout-grid-type:,left:,
        letter-spacing:,line-break:,line-height:,list-style:,
        list-style-image:,list-style-position:,list-style-type:,margin:,
        margin-bottom:,margin-left:,margin-right:,margin-top:,
        marker-offset:,marks:,max-height:,max-width:,min-height:,
        min-width:,transition-duration:,transition-property:,
        transition-timing-function:,transform:,
        -moz-transform:,-moz-binding:,-moz-border-radius:,
        -moz-border-radius-topleft:,-moz-border-radius-topright:,
        -moz-border-radius-bottomright:,-moz-border-radius-bottomleft:,
        -moz-border-top-colors:,-moz-border-right-colors:,
        -moz-border-bottom-colors:,-moz-border-left-colors:,-moz-opacity:,
        -moz-outline:,-moz-outline-color:,-moz-outline-style:,
        -moz-outline-width:,-moz-user-focus:,-moz-user-input:,
        -moz-user-modify:,-moz-user-select:,orphans:,outline:,
        outline-color:,outline-style:,outline-width:,overflow:,
        overflow-X:,overflow-Y:,padding:,padding-bottom:,padding-left:,
        padding-right:,padding-top:,page:,page-break-after:,
        page-break-before:,page-break-inside:,pause:,pause-after:,
        pause-before:,pitch:,pitch-range:,play-during:,position:,quotes:,
        -replace:,richness:,right:,ruby-align:,ruby-overhang:,
        ruby-position:,-set-link-source:,size:,speak:,speak-header:,
        speak-numeral:,speak-punctuation:,speech-rate:,stress:,
        scrollbar-arrow-color:,scrollbar-base-color:,
        scrollbar-dark-shadow-color:,scrollbar-face-color:,
        scrollbar-highlight-color:,scrollbar-shadow-color:,
        scrollbar-3d-light-color:,scrollbar-track-color:,table-layout:,
        text-align:,text-align-last:,text-decoration:,text-indent:,
        text-justify:,text-overflow:,text-shadow:,text-transform:,
        text-autospace:,text-kashida-space:,text-underline-position:,top:,
        unicode-bidi:,-use-link-source:,vertical-align:,visibility:,
        voice-family:,volume:,white-space:,widows:,width:,word-break:,
        word-spacing:,word-wrap:,writing-mode:,z-index:,zoom:
        },
        morecomment=[s]{<!--}{-->},
        tag=[s]
}
