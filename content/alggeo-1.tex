\section{Classical Algebraic Geometry}

\begin{frame}{Foray Into Algebraic Geometry}
    \begin{block}{Projective Space}
    Playing field is $n$\emph{-dimensional projective space}, $\PP^{n}$:
        $$ \PP^{n} := \{ (z_{0}, \ldots, z_{n}) \in \CC^{n} \} / (\vb{x} \sim \lambda \cdot \vb{y}), \quad \lambda \neq 0, $$
    that is, its elements consists of \emph{lines through the origin} in $\CC^{n}$.
    \end{block}
    
    TODO: FIGURE
\end{frame}

\begin{frame}{Varieties}
    
    \emph{Varieties} are the geometric studied in algebraic geometry, and are the \emph{vanishing sets}\footnote{from `\emph{Verschwindungsmenge}'} for a system of polynomials.

    TODO: FIGURES

\end{frame}

\begin{frame}{Segre Varieties}
        \emph{Segre varieties} come from $\sigma: \PP^{n} \times \PP^{m} \ra \PP^{(n+1)(m+1) - 1}$, that sends $([X],[Y])$ to the pairwise products of their components:
            $$ \sigma : ([X_{1}, \ldots, X_{n+1}], [Y_{1}, \ldots, Y_{m+1}]) \mapsto [\ldots, X_{i}Y_{j}, \ldots ], $$

        \begin{block}{Example}
        $$\sigma : \PP^{1} \times \PP^{1} \ra \PP^{3},\ ([X_{1}, X_{2}], [Y_{1}, Y_{2}]) \mapsto [ X_{1}Y_{1}, X_{1}Y_{2}, X_{2}Y_{1}, X_{2}Y_{2} ].  $$
        
        Set $[ X_{1}Y_{1}, X_{1}Y_{2}, X_{2}Y_{1}, X_{2}Y_{2} ] = [p_{11}, p_{12}, p_{21}, p_{22}]$,
        $$ \rightsquigarrow \det \begin{pmatrix} p_{11} & p_{12} \\ p_{21} & p_{22} \end{pmatrix} = 0 \iff \rank \begin{pmatrix} p_{11} & p_{12} \\ p_{21} & p_{22} \end{pmatrix} \leq 1. $$
        \end{block}
\end{frame}

\begin{frame}
    
    \begin{block}{Rulings}
    $\sigma(\PP^{1} \times \PP^{1}) = \{ [p_{11}, p_{12}; p_{21}, p_{22}] : \det (p_{ij}) = 0 \} $ is an example of a \emph{determinantal variety}.
    
    This example has two families of lines inside of it; the images of $\sigma( [p_{11}, p_{12}] \times \{ Q \} )$ and $\sigma( \{ Q \} \times [p_{21}, p_{22}])$, which are called \emph{rulings of the surface}.
    \end{block}

\end{frame}

\begin{frame}{Manifold of Independence}

    \begin{itemize}
        \item Let $\Delta \subset \RR^{4}$ be the tetrahedron with vertices given by the four basis vectors, $A_{i} = e_{i}$, and let a general point $p = (p_{ij})$ inside of $\Delta$ be represented by
            \begin{center}
        \begin{table}[]
        $p_{ij} = (p_{11}, p_{12}, p_{21}, p_{22}) =$ 
        \begin{tabular}{|l|l|}
        \hline
        $p_{11}$ & $p_{12}$ \\ \hline
        $p_{21}$ & $p_{22}$ \\ \hline
        \end{tabular}
        \end{table}
        \end{center}

        \item Fienberg and Gilbert have shown that the two rulings are given by
    \begin{center}
    \begin{table}[]
    \begin{tabular}{|c|c|}
    \hline
    $st$ & $s(1-t)$ \\ \hline
    $t(1-s)$ & $(1-s)(1-t)$ \\ \hline
    \end{tabular}
    $\quad (0 \leq s,t \leq 1),$
    \end{table}
    \end{center}
    
    which is a hyperbolic paraboloid inside of $\Delta$.
    
    \item They call this the \emph{manifold of independence}; any point on this surface has independent row and column marginal totals.

    \end{itemize}
    
\end{frame}

\begin{frame}
    TODO: FIGURE.
\end{frame}